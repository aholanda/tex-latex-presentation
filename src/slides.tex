\documentclass{beamer}

% Setup appearance:
\usefonttheme{structurebold}

\usetheme{Darmstadt}
\setbeamerfont*{frametitle}{size=\normalsize,series=\bfseries}
\setbeamertemplate{navigation symbols}{}

\usepackage[brazilian]{babel}

% additional packages
\usepackage{hyperref}
\usepackage{ifpdf}
\usepackage{pgf,pgfarrows,pgfnodes,pgffor,pgfkeys} % to draw figures
\usepackage{tikz}

\colorlet{redshaded}{red!25!bg}
\colorlet{shaded}{black!25!bg}
\colorlet{shadedshaded}{black!10!bg}
\colorlet{blackshaded}{black!40!bg}

\colorlet{darkred}{red!80!black}
\colorlet{darkblue}{blue!80!black}
\colorlet{darkgreen}{green!80!black}

%%%%%%%%%%%%%%%%%%%% IMAGES
\def\imageheight{2cm}
\def\imagewidth{2.25cm}
\def\textimagewidth{4cm}

%% photos
\pgfdeclareimage[interpolate=true,width=\imagewidth]{knuth}{img/donknuth.png}
\pgfdeclareimage[interpolate=true,width=\imagewidth]{lamport}{img/leslie.png}
% codes
\pgfdeclareimage[interpolate=true,height=\imageheight]{texpage}{img/textext.png}
\pgfdeclareimage[interpolate=true,height=\imageheight]{dvicode}{img/dvicode.png}
\pgfdeclareimage[interpolate=true,height=\imageheight]{pscode}{img/pscode.png}
\pgfdeclareimage[interpolate=true,height=\imageheight]{pspage}{img/pspage.png}
%% printers imgs
\pgfdeclareimage[interpolate=true,height=\imageheight]{laserprinter}{img/laserprinter.png}
\pgfdeclareimage[interpolate=true,height=\imageheight]{colorprinter}{img/colorlaserprinter.png}
%% Letter
\pgfdeclareimage[interpolate=true,height=3cm]{glyph}{img/gglyph.png}
\pgfdeclareimage[interpolate=true,height=4cm]{gletter}{img/g.png}
\pgfdeclareimage[interpolate=true,height=4cm]{hletter}{img/h.png}
\pgfdeclareimage[interpolate=true,height=6.5cm]{parenthesis}{img/parenthesis.png}
\pgfdeclareimage[interpolate=true,height=4cm]{gletterwhite}{img/gWhite.png}
\pgfdeclareimage[interpolate=true,height=4cm]{hletterwhite}{img/hWhite.png}
\pgfdeclareimage[interpolate=true,height=6.5cm]{parenthesiswhite}{img/parenthesisWhite.png}


\title{\Large{\TeX: typesetting system \\ \LaTeX: document preparation system}}
  
\author{Adriano de Jesus Holanda}
\institute[]{Departamento de F\'i�sica e Matem\'atica\\
FFCLRP-USP}

\date{2010-04-28}
  
\begin{document}
% Definitions
\def\imgheight{8} % default images height
\def\imgwidth{6} % default images width
\def\bs{$\backslash$}

\frame{\titlepage}

\part{Main Part}
\frame{\frametitle{Outline}\tableofcontents[part=1]}

\section{\TeX}

\subsection{Introduction}

\begin{frame}

  \frametitle{\TeX\ -- {\large $\tau \epsilon \chi$}\ {\tiny (greek: ``art/craft'')}}
  \framesubtitle{Typesetting system}

  \begin{columns}
    
    \begin{column}{2.5cm}
      \def\x{-.3cm}
      \def\y{-1cm}
        \pgfuseimage{knuth}
    \end{column}
    
    \begin{column}{7.5cm}  

      \begin{block}{Prof. Donald Erwin Knuth}
        \begin{itemize}
        \item Emeritus Professor of ``The Art of Computer Programming'' (TAOCP)
        \item Stanford University, USA
        \item Main contributor to the {\bf Analysis of Algorithms} field using asymptotic notation
        
        \end{itemize}
      \end{block}
  
    \begin{alertblock}{Motivation}
      \begin{itemize}
      \item Unsatisfaction with printed version of TAOCP2 2nd ed.,
        due to declined quality of the typesetting using new software
        and digital printer technology.
      \end{itemize}
    \end{alertblock}
    \end{column}
  \end{columns}
\end{frame}


\begin{frame}
  \frametitle{\TeX family}
   \begin{list}
      {\color{blue!90!bg}{$\star$}}{\setlength{\parsep}{.5cm}}
    \item {\em Language} $\rightarrow$ \TeX~(Pascal);
    \item {\em Font engine} $\rightarrow$ METAFONT;
    \item {\em Font family} $\rightarrow$ Computer Modern;
    \item {\em Output format} $\rightarrow$ Device independent format (DVI) [Fuchs, 1979].
   \end{list}
\end{frame}

% measures used to draw rectangles simbolizing
% pages of manuscripts
\def\xptex{2.5} % x position of TeX page
\def\xshift{3} % x position of DVI page

\begin{frame}

  \frametitle{Processing steps}
  
  \begin{tikzpicture}
    
    \begin{pgfpicture}{-0.5cm}{0cm}{4cm}{6cm}

    \pgfputat{\pgfxy(0,2)}{\pgfbox[center, center]{\pgfuseimage{texpage}}}
      \only<2-5,6-8> {
        \pgfputat{\pgfxy(4.25, 2)}{\pgfbox[center, center]{\pgfuseimage{dvicode}}}
      }
      \only<3-4>{
        \pgfputat{\pgfxy(8.5, 4.5)}{\pgfbox[center, center]{\pgfuseimage{laserprinter}}}
      }
      \only<7>{
        \pgfputat{\pgfxy(8.5, -.5)}{\pgfbox[center, center]{\pgfuseimage{laserprinter}}}
      }
      \only<4,8>{
        \pgfputat{\pgfxy(8.5, -.5)}{\pgfbox[center, center]{\pgfuseimage{colorprinter}}}
      }
      \only<5>{
        \pgfputat{\pgfxy(8.5, 2)}{\pgfbox[center, center]{\pgfuseimage{pscode}}}
      }
      \only<6-8>{
        \pgfputat{\pgfxy(8.5, 2)}{\pgfbox[center, center]{\pgfuseimage{pspage}}}
      }
      
      %% ARROWS
      \pgfsetendarrow{\pgfarrowto}
      \only<2-5,6-8>{
        \pgfxyline(1.35, 2.75)(3.25, 2.75)
      }
      \only<3-4>{
        \pgfxyline(5.2, 2.75)(7.55, 5)
        \pgftext[base,at={\pgfxy(7.25, 3.5), rotate=0}] {X converter}
      }
      \only<4>{
        \pgfxyline(5.2, 2.75)(7.5, 1.1)
        \pgftext[base,at={\pgfxy(7.15, 2), rotate=0}] {Y converter}
      }
      %% TEXT
      \pgfputat{\pgfxy(.5, 3.25)}{\pgfbox[left,bottom]{\color{red!20!}TeX}}
      \pgfputat{\pgfxy(-1.65, 4.25)}{\pgfbox[left,bottom]{TeX [Knuth, 1978]}}
      
      \only<2-5,6-8>{
        \pgfputat{\pgfxy(2.75, 4.25)}{\pgfbox[left,bottom]{DVI [Fuchs, 1979]}}
        \pgfputat{\pgfxy(2.3, 2.8)}{\pgfbox[center, bottom]{tex}}
      }
      \only<5-8>{
        \pgfputat{\pgfxy(6.5, 4.25)}{\pgfbox[left,bottom]{PostScript [Adobe, 1982]}}
      }
      \only<6-8>{
        \pgfxyline(5.1, 2.75)(6.9, 2.75)
        \pgfputat{\pgfxy(5.5, 2.8)}{\pgfbox[left,bottom]{dvips}}
      }
      \only<7,8>{
        \pgfxyline(8.5, 2)(8.5, 1.4)
      }

    \end{pgfpicture}
       
  \end{tikzpicture}

\end{frame}

\begin{frame}
	\frametitle{Typographic concepts}
	
	\begin{itemize}
		\item<1-| uncover@1 > {\bfseries Unit of measure}: PostScript \alert{\bf point} $\equiv \frac{1}{72} in \cong 0.3528 mm$;
		\item<2-| uncover@2> {\bfseries Glyph}: pictorial representation of typographical token;
		\item<3-| uncover@3,4> {\bfseries Kerning}: Shifting between pair of consecutive glyphs.
	\end{itemize}
	
	\vfill
          \only<1>{
	
        \begin{tikzpicture}
            % INCH
            \def\width{4}
            \def\scale{0.2}
            \draw[-] (0in,0in) -- (0in,-\scale in);
            \foreach \x in {0,...,3} {
              \draw[-] (0in,0in) -- (\x in,0in);
              \draw[-] (\x in,0in) -- (\x in,-\scale in);
              \draw[-] (\x.5 in,0in) -- (\x.5 in,-\scale *0.5 in);
              % 
              \draw[-] (\x.25 in,0in) -- ( \x.25  in,-\scale *0.25 in);
              \draw[-] (\x.75 in,0in) -- ( \x.75  in,-\scale *0.25 in);
              
            } % end of foreach x
            \draw[-] (0in,0in) -- (\width in, 0in) node[right] {\width in};
            \draw[-] (\width in,0in) -- (\width in,-\scale in);
        \end{tikzpicture}
        
        \vspace{1cm}
        
        \begin{tikzpicture}
            % POINT
            \def\width{300}
            \def\scale{15}
            \draw[-] (0 pt,0 pt) -- (0 pt,-\scale pt);
            \foreach \x in {0, 1, 2} {
              \draw[-] (0 pt,0 pt) -- (\x00 pt, 0 pt);
              \draw[-] (\x00 pt, 0pt) -- (\x00  pt, -\scale pt);
              \draw[-] (\x50 pt, 0 pt) -- (\x50 pt, -\scale *0.5 pt);
              % 
              \draw[-] (\x25 pt, 0 pt) -- ( \x25 pt, -\scale *0.25 pt);
              \draw[-] (\x75 pt, 0 pt) -- ( \x75 pt, -\scale *0.25 pt);
              
            } % end of foreach x
            \draw[-] (0 pt, 0 pt) -- (\width pt, 0 pt) node[right] {\width pt};
            \draw[-] (\width pt, 0 pt) -- (\width pt, -\scale pt);

        \end{tikzpicture}

        \vspace{1cm}

        % CENTIMETER
        \begin{tikzpicture}
          \def\width{10}
            \def\scale{0.5}
            \draw[-] (0 cm,0 cm) -- (0 cm,-\scale cm);
            \foreach \x in {0,...,9} {
              \draw[-] (0 cm,0 cm) -- (\x cm, 0 cm);
              \draw[-] (\x cm, 0cm) -- (\x  cm, -\scale cm);
              \draw[-] (\x.5 cm, 0 cm) -- (\x.5 cm, -\scale *0.5 cm);
              % 
              %\draw[-] (\x25 cm, 0 cm) -- ( \x25 cm, -\scale *0.25 cm);
  %            \draw[-] (\x75 cm, 0 cm) -- ( \x75 cm, -\scale *0.25 cm);
              
            } % end of foreach x
            \draw[-] (0 cm, 0 cm) -- (\width cm, 0 cm) node[right] {\width cm};
            \draw[-] (\width cm, 0 cm) -- (\width cm, -\scale cm);

        \end{tikzpicture}
        
          } % end of only
          
        \begin{tikzpicture}%
	  \only<2> {
	    \pgfputat{\pgfxy(4, 0.5)}{\pgfbox[left,center]{\pgfuseimage{glyph}}}%
	  }
        \end{tikzpicture}%

	
	% positive kerning
	\begin{tikzpicture}[scale=0.5]
          \only<3>{
	    \draw (0, 0) rectangle (4,4);
	    \draw (5, -0.5) rectangle (9, 3.5);
	    \draw[<->] (4, 1.5) -- (5, 1.5) node[above,midway] {+};
	  } % end of only
        \end{tikzpicture}

	% negative kerning
	\begin{tikzpicture}[scale=0.5]
          \only<4>{
	    \draw (0, -1) rectangle (4,3);
	    \draw (3, -1.5) rectangle (8, 2.5);
	    \filldraw[fill=green!20!white] (3, -1) rectangle (4, 2.5);
	    \draw[<->] (3, 0.5) -- (4, 0.5) node[above,midway] {-};
	  } % end of only
	\end{tikzpicture}

\end{frame}

\subsection{Basics}

\begin{frame}
  \frametitle{Boxes}
  \begin{columns}
	\begin{column}{3cm}
		\begin{tikzpicture}[scale=.6]
			\def\width{5} % width of the rectangles
			\def\height{8}
			\def\depth{2}
			\def\sp{.35} % space
			\def\lsp{.275} % litle space
			
			\draw[<->] (\width + 2*\lsp + \sp, 0) -- (\width + 2*\lsp + \sp, \height) node[midway,fill=white] {height}; %  |^
			\draw (0, 0) rectangle (\width, \height); %2nd rectangle
			\filldraw [red] (0, 0) circle (2pt); % point
			\draw[->] (-\height/2, 0) -- (0, 0)  node[midway,above] {Reference} node[midway,below] {point}; % Reference point ->
			\draw[<->] (\width + 2*\lsp + \sp, -\depth) -- (  \width + 2*\lsp + \sp, 0) node[midway,fill=white] {depth} ; %  |^
			\draw (0, 0) -- (\width, 0) node[midway, above] {Baseline};
			\draw (0, -\depth) rectangle (\width, 0); %1st base rectangle
			\draw[<->] (0, -\depth -\sp -\lsp) -- ( \width ,-\depth -\sp -\lsp) node[fill=white,midway] {width}; % <->
                        \only<2>{
                          % axis
                          \draw[red, very thick, ->] (0, 0) -- (\width/2, 0);
                          \draw[red,very thick, ->] (0, 0) -- (0, \height/2);
                          \draw[red, very thick, ->] (0, 0) -- (0, -\depth/2);
                          \filldraw [red] (0, 0) circle (4pt); % point
                        }
		\end{tikzpicture}
	\end{column}
\end{columns}
\end{frame}

\begin{frame}
  \frametitle{Boxes}
  \framesubtitle{Examples}

	\pgfputat{\pgfxy(0.5, 2)}{\pgfbox[left,center]{{\bf \alert{10 pt}} characters} }
	\vspace{1cm}
	\begin{tikzpicture}
	  \only<1>{
	    \pgfputat{\pgfxy(-1, -.9)}{\pgfbox[left,center]{\pgfuseimage{gletter}}}
	  }
	  \only<2->{
	    \pgfputat{\pgfxy(-1, -.9)}{\pgfbox[left,center]{\pgfuseimage{gletterwhite}}}
	  }
	  \only<2>{
	    \pgfputat{\pgfxy(3, -.6)}{\pgfbox[left,center]{\pgfuseimage{hletter}}}
	  }
	  \only<3>{
	    \pgfputat{\pgfxy(3, -.6)}{\pgfbox[left,center]{\pgfuseimage{hletterwhite}}}
	    \pgfputat{\pgfxy(7, 0)}{\pgfbox[left,center]{\pgfuseimage{parenthesis}}}
	  }
	\end{tikzpicture}
\end{frame}

% some definitions of measures applied to boxes
% used in the ``glue'' examples
%% box 1
\pgfkeyssetvalue{/width/1}{5}
\pgfkeyssetvalue{/depth/1}{5}
\pgfkeyssetvalue{/height/1}{5}
%% box 2
\pgfkeyssetvalue{/width/2}{6}
\pgfkeyssetvalue{/depth/2}{1}
\pgfkeyssetvalue{/height/2}{1}
%% box 3
\pgfkeyssetvalue{/width/3}{3}
\pgfkeyssetvalue{/depth/3}{1.5}
\pgfkeyssetvalue{/height/3}{6}
%% box 4
\pgfkeyssetvalue{/width/4}{8}
\pgfkeyssetvalue{/depth/4}{2}
\pgfkeyssetvalue{/height/4}{3.5}
% spaces
\pgfkeyssetvalue{/space/1}{9} % first space
\pgfkeyssetvalue{/space/2}{9} % second space
\pgfkeyssetvalue{/space/3}{12} % third space
\pgfkeyssetvalue{/space/4}{0} % there is nothing after box 4
\newcounter{totalspace} % store toral space
\setcounter{totalspace}{0}
\foreach \i in {1,2,3,4} {
  \addtocounter{totalspace}{\pgfkeysvalueof{/space/\i}}
}
\pgfkeyssetvalue{/total/space}{\arabic{totalspace}}
% stretches
\pgfkeyssetvalue{/stretch/1}{3} % first stretch
\pgfkeyssetvalue{/stretch/2}{6} % second stretch
\pgfkeyssetvalue{/stretch/3}{0} % third stretch
\pgfkeyssetvalue{/stretch/4}{0} % DONT
\newcounter{totalstretch} % store total stretch
\setcounter{totalstretch}{0}
\foreach \i in {1,2,3,4} {
  \addtocounter{totalstretch}{\pgfkeysvalueof{/stretch/\i}}
}
\pgfkeyssetvalue{/total/stretch}{\arabic{totalstretch}}
% shrinks
\pgfkeyssetvalue{/shrink/1}{1} % first shrinks
\pgfkeyssetvalue{/shrink/2}{2} % second shrinks
\pgfkeyssetvalue{/shrink/3}{0} % third shrinks
\pgfkeyssetvalue{/shrink/4}{0} % DONT
\newcounter{totalshrink} % store total shrink
\setcounter{totalshrink}{3} % kick the boot
\pgfkeyssetvalue{/total/shrink}{\arabic{totalshrink}}

% define original baseline reset for the spaces
\pgfkeyssetvalue{/xbaseline/1}{0}
\pgfkeyssetvalue{/xbaseline/2}{14}
\pgfkeyssetvalue{/xbaseline/3}{29}
\pgfkeyssetvalue{/xbaseline/4}{44}

% counters
\newcounter{next}% to go one step further
\newcounter{totalwidth} % to count total width
\newcounter{totaladjust} % sumation of adjusts
 \newcounter{xbase} % x base to start boxes

\def\BOXES{
  % reset baseline
  % 4 BOXES
  \foreach \i in {1, 2, 3, 4} { 
    \foreach \x/\y/\X/\Y/\width in {
      \pgfkeysvalueof{/xbaseline/\i}/
      -\pgfkeysvalueof{/depth/\i}/
      \pgfkeysvalueof{/xbaseline/\i}+\pgfkeysvalueof{/width/\i}/
      \pgfkeysvalueof{/height/\i}/
      \pgfkeysvalueof{/width/\i}} {
      % {
      \draw (\x,\y) rectangle (\X,\Y) node[above] {width \width};
      \draw (\x, 0) -- (\X, 0); % baselines
    }
  }        
  %% spaces and text
  \setcounter{totalwidth}{0}
  \setcounter{next}{1}
  \foreach \idx in {1,2,3} {
    \addtocounter{next}{1}
    \foreach \x/\y/\X/\Y/\sp/\st/\sh in {
      \pgfkeysvalueof{/width/\idx}/%
      0/%
      \pgfkeysvalueof{/xbaseline/\arabic{next}}/%
      0/%
      \pgfkeysvalueof{/space/\idx}/%
      \pgfkeysvalueof{/stretch/\idx}/%
      \pgfkeysvalueof{/shrink/\idx}} {
      %{
      \draw[dashed] (\pgfkeysvalueof{/xbaseline/\idx} +\x, \y) -- (\X, \Y);
      \draw (\pgfkeysvalueof{/xbaseline/\idx} + \x+\sp/2,\y-.1) node[below] {space \sp};
      \draw (\pgfkeysvalueof{/xbaseline/\idx} + \x+\sp/2,\y-2.1) node[below] {stretch \st};
      \draw (\pgfkeysvalueof{/xbaseline/\idx} + \x+\sp/2,\y-4) node[below] {shrink \sh};
      \draw[<->] (0,-9) -- (52, -9) node[fill=white,midway] {width 52};
    } % end of inner foreach
  } % end of outer foreach
}

\def\ADJUSTEDLISTOFBOXES #1 #2 { %
  \setcounter{xbase}{0}%
  \setcounter{totalwidth}{0}%
  % rectangles
  \foreach \i in {1,2,3,4} {%
    % {
    % FOREACH
    \foreach \x/\y/\X/\Y/\width/\sp/\st/\sh/\adj in {
      %{
      \arabic{xbase}/
      -\pgfkeysvalueof{/depth/\i}/
      \arabic{xbase}+\pgfkeysvalueof{/width/\i}/
      \pgfkeysvalueof{/height/\i}/
      \pgfkeysvalueof{/width/\i}/
      \pgfkeysvalueof{/space/\i}/
      \pgfkeysvalueof{/stretch/\i}/
      \pgfkeysvalueof{/shrink/\i}/
      \pgfkeysvalueof{/adjust/\i}} {
        % {
        \draw (\x,\y) rectangle (\X,\Y); % box
        \draw (\x, 0) -- (\X, 0); % baseline
        \fill [color=red]  (\x, 0) circle (2pt); % point
        \draw[dashed] (\X, 0) -- (\X + \sp + \adj, 0);
        \draw[dashed] (\x, 0) -- (\X, 0);
        
        % defines the signal of the adjust
        \ifnum#2=51\def\signal{-1}\def\action{\sh}\else\def\signal{+6}\def\action{\st}\fi
        {
          \ifnum\i<4 % after 4 do not have space
          \draw (\x+ + \width + \sp/2, -5.1) node[below] {\tiny $\sp +(\signal/\pgfkeysvalueof{/total/#1})x\action=$}
          (\x + \width + \sp/2, -9.1) node[below] {\small $\sp \adj$};
          \fi
        }
      } % END of FOREACH \x...
      \addtocounter{xbase}{\pgfkeysvalueof{/width/\i}} % add width
      \addtocounter{xbase}{\pgfkeysvalueof{/space/\i}} % add old xbase value
      \addtocounter{xbase}{\pgfkeysvalueof{/adjust/\i}} % add old xbase value

    } % END of FOREACH{1,2,3,4}
  \draw[<->] (0,-13) -- (#2, -13) node[fill=white,midway] {width #2};
} % END of def\ADJUSTED...

\def\pleasetex#1{
  \draw (0,15) node[right] {``\TeX! Please, make the horizontal list into a box that is {\color{darkred}#1 units wide}''};
  \draw[<->,color=red] (0,-13) -- (#1, -13) node[fill=white,midway] {width #1};
}
   
\begin{frame}
  \frametitle{Glue}
  \framesubtitle{Stretchability example}
  \begin{tikzpicture}[scale=.2]
   
    \BOXES

    \addtocounter{totalwidth}{\pgfkeysvalueof{/width/4}} % add box 4 width

    \only<2>{
     \pleasetex{58}
    }
  \end{tikzpicture}
\end{frame}


%%%%%%%%%%%%%% 58 units slide
%%% TeX obey the ask to make horizontal box of 58 units

\begin{frame}

  \frametitle{Glue}
  \framesubtitle{Stretchability example}
  \begin{tikzpicture}[scale=.2]

    %% adjust to do
    \pgfkeyssetvalue{/adjust/1}{+2}
    \pgfkeyssetvalue{/adjust/2}{+4}
    \pgfkeyssetvalue{/adjust/3}{+0}
    \pgfkeyssetvalue{/adjust/4}{+0} % there is no adjust after 4
    \ADJUSTEDLISTOFBOXES stretch  58
  \end{tikzpicture}

\end{frame}

%%%%%%%%%%%%%%%%%%%% SHRINK
\begin{frame}
  \frametitle{Glue}
  \framesubtitle{Shrinkability example}
  \begin{tikzpicture}[scale=.2]

    \BOXES
   
    \only<2> {
      \pleasetex{51}
    }
  \end{tikzpicture}

\end{frame}


\begin{frame}
  \frametitle{Glue}
  \framesubtitle{Shrinkability example}

  \begin{tikzpicture}[scale=.2]
    %% adjust to do
    \pgfkeyssetvalue{/adjust/1}{-0.33}
    \pgfkeyssetvalue{/adjust/2}{-0.66}
    \pgfkeyssetvalue{/adjust/3}{-0}
    \pgfkeyssetvalue{/adjust/4}{-0} % there is no adjust after 4
    \setcounter{totaladjust}{0}
    \foreach \i in {1,2,3,4} {
      \addtocounter{totaladjust}{\pgfkeysvalueof{/adjust/\i}{0}}
    }
    
    \ADJUSTEDLISTOFBOXES shrink 51
  \end{tikzpicture}

\end{frame}

\begin{frame}
	\frametitle{Vertical boxes}
	
	\def\scalevalue{0.35}
	\begin{tikzpicture}[scale=\scalevalue]
		
		\invisible<7->{
			\draw (0, -1.5) rectangle (4, 5);
		}
		\only<7->{
			\draw (0, -1.5) rectangle (4, 8);
			\draw[<->,color=red!70!black] (-1, 0) -- (-1, 8) node[midway,left] {8pt};
			\invisible<8>{
				\draw[<->,color=red!40!black] (-1, 10) -- (-1, 11) node[midway,left] {\tiny 1pt};
			}
		}
		\draw[dashed] (-4, 0) -- (7, 0);
		
		\only<1-4>{
			\draw (0, 11) rectangle (5, 18);
			\draw[dashed] (-4,  14) -- (7, 14);
		}
		
		\only<5-7>{
			\draw (0, 13) rectangle (5, 20);
			\draw[dashed] (-4,  16) -- (7, 16);
		}
		
		\only<2->{
			\invisible<5->{
				\draw[<->] (6,  0) -- (6 , 14);
			}
		}
		
		\only<2>{
			 \pgfputat{\pgfxy(7*\scalevalue, 7*\scalevalue)}
				{\pgfbox[left, center]{\tt \small \bs baselineskip=$\langle$glue$\rangle$}}
			\pgfputat{\pgfxy(7*\scalevalue,  5.75*\scalevalue)}
				{\pgfbox[left, center]{\tt \small \bs lineskip=$\langle$glue$\rangle$}}
			\pgfputat{\pgfxy(7*\scalevalue, 4.5*\scalevalue)}
				{\pgfbox[left, center]{\tt \small \bs lineskiplimit=$\langle$dimen$\rangle$}}
		}
		
		\only<3->{
			 \pgfputat{\pgfxy(7*\scalevalue, 7*\scalevalue)}
				{\pgfbox[left, center]{\tt \small \alert<4-6>{\bs baselineskip=12pt plus 2pt}}}
			\pgfputat{\pgfxy(7*\scalevalue,  5.75*\scalevalue)}
				{\pgfbox[left, center]{\tt \small \alert<8>{\bs lineskip=3pt minus 1pt}}}
			\pgfputat{\pgfxy(7*\scalevalue, 4.5*\scalevalue)}
				{\pgfbox[left, center]{\tt \small \alert<7>{\bs lineskiplimit=2pt}}}
		}
		
		\only<3->{	
			\invisible<7->{
				\draw[<->] (-1, 0) -- (-1, 5) node[midway,left] {\small 5pt};
			}
			\invisible<5->{
				\draw[<->] (-1, 11) -- (-1, 14) node[midway,left] {\small 3pt};
			}
		}
		\only<4,5>{
			\draw[<->,color=red] (-1,  5) -- (-1, 11) node[midway,left] {\small 4pt};
		}
		
		\only<5-7>{
			\draw[<->] (6,  0) -- (6 , 16);
			\draw[<->] (-1, 13) -- (-1, 16) node[midway,left] {\small 3pt};
		}
		\only<6>{
			\draw[<->,color=red] (-1,  5) -- (-1, 13) node[midway,left] {\small 4pt + 2pt = 6pt};
		}
		
		\only<8>{
			\draw[<->,color=red] (-1,  8) -- (-1, 10) node[midway,left] {\tiny 3pt - 1pt = 2pt};
			\draw (0, 10) rectangle (5, 17);
			\draw[dashed] (-4,  13) -- (7, 13);
			\draw[<->] (6,  0) -- (6 , 13);
			\draw[<->] (-1, 10) -- (-1, 13) node[midway,left] {\small 3pt};
		}
	\end{tikzpicture}
	
\end{frame}

\subsection{Running \TeX}

\begin{frame}
	\frametitle{Running \TeX (1)}
	\framesubtitle{Setting horizontal size}

        \begin{tikzpicture}
          \draw (0, 0) node[fill=black,text width=10cm,text justified]{
            \tt \small
            \color{white}
            \$ tex \\
            \$ **\bs relax \\
          \$ *\bs input story \\
          \$ *\bs hsize=3in \bs input story \\
          \$ *\bs hsize=2.5in \bs input story \\
          \$ *\bs hsize=2in \bs input story \\
          \$ *\bs end \\
          \$ xdvi texput.dvi 
      };
        \end{tikzpicture}
\end{frame}

\begin{frame}
  \frametitle{Running \TeX (2)}
  \framesubtitle{Setting tolerance}

  \begin{tikzpicture}
    \draw (0, 0) node[fill=black,text width=10cm,text justified]{
      \tt \small
      \color{white}
      \$ tex \\
      \$ **\bs relax \\
      \$ *\bs input story \\
      \$ *\bs hsize=2in \bs tolerance=1600 \bs input story \\
      \$ *\bs hsize=1.5in \bs input story \\
      \$ *\bs tolerance=10000 \bs input story  \\
      \$ *\bs hsize=.75in \bs input story \\
      \$ *\bs end \\
      \$ xdvi texput.dvi 
    };
  \end{tikzpicture}
	
\end{frame}

\subsection{Ho\texorpdfstring{$\omega$}{w} \texorpdfstring{$\alpha$}{a}bout M\texorpdfstring{$\alpha$}{a}th?}

\begin{frame}
    \frametitle{\$M$\alpha$th symbols\$}
  
    \begin{columns}

    \begin{column}{4cm}      
    \begin{block}{greek letters}
        \begin{tabbing}
            {\em Input} \hspace{1.8cm} \= {\em Output} \\
            \$\bs alpha\$ \> $\alpha$ \\
            \$\bs gamma\$ \> $\gamma$ \\
            \$\bs Gamma\$ \> $\Gamma$ \\
            \$\bs Psi\$ \> $\Psi$ \\
            \$\bs Omega\$ \> $\Omega$ \\
    \end{tabbing}
    \end{block}
    \end{column}

    \only<2>{
      \begin{column}{4cm}      
        \begin{block}{operators}
          \begin{tabbing}
            {\em Input} \hspace{1.8cm} \= {\em Output} \\
            \$\bs sum\$ \> $\sum$ \\
            \$\bs prod\$ \> $\prod$ \\
            \$\bs int\$ \> $\int$ \\
            \$\bs cap\$ \> $\cap$ \\
            \$\bs vee or \bs lor\$ \> $\vee$ \\
          \end{tabbing}
        \end{block}
      \end{column}
    }
    
    \end{columns}

\end{frame}

\def\metahat{$\hat{\ }$}
\begin{frame}
	\frametitle{\$M$\alpha$th formul$\alpha$s\$}
	
	\begin{columns}
	
	\begin{column}{4cm}
	\begin{tabbing}
		{\em Input} \hspace{2.5cm} \= {\em Output} \\
		{\tt \$x\metahat 2\$} \> $x^2$\\
		{\tt \$x\_2\$} \> $x_2$\\
		{\tt \$2\metahat x\$} \> $2^x$\\
		{\tt \$x\metahat 2y\metahat 2\$} \> $x^2y^2$ \\
		{\tt \$x \metahat  2y \metahat  2\$} \> $x ^ 2y ^ 2$ \\
		{\tt \$x\_2y\_2\$} \> $x_2y_2$ \\
		{\tt \$\_2F\_3\$} \> $_2F_3$ \\
	\end{tabbing}
	\end{column}
	
        \begin{column}{4cm}
            \begin{tabbing}
		{\em Input} \hspace{2.3cm} \= {\em Output} \\
                {\tt \$x\metahat \{2y\}\$} \> $x^{2y}$ \\
                {\tt \$2\metahat \{2\metahat x\}\$} \> $2^{2^x}$ \\
                {\tt \$2$\hat{\ }\{$2\metahat \{2\metahat x\}\}\$} \> $2^{2^{2^x}}$ \\
                {\tt \$y\_\{x\_2\}\$} \> $y_{x_2}$ \\
                {\tt \$y\_\{x\metahat 2\}\$} \> $y_{x^2}$ \\
            \end{tabbing}        
        \end{column}
        
    \end{columns}	

\end{frame}

\begin{frame}
    \frametitle{$+$ \only<2>{$+$} \$\$ M$\alpha$th \$\$}
    
    \only<1>{
{\small   \$\$ \bs sqrt\{1+ \bs sqrt\{1+\bs sqrt\{1+\bs sqrt\{1+\bs sqrt\{1+x\}\}\}\}\} \bs eqno(1) \$\$    }
    $$ \sqrt{1+\sqrt{1+\sqrt{1+\sqrt{1+\sqrt{1+x}}}}} \eqno(1) $$ 
    }

    \only<2>{
        $$a_0+{1\over\displaystyle a_1+
            { 1\over\displaystyle a_2+
                { 1\over\displaystyle a_3+
                     { 1\over a_4}}}} \eqno(2)$$
     }
\end{frame}

\section{\LaTeX}

\subsection{Introduction}

\begin{frame}
  \frametitle{\LaTeX -- Document preparation system}
  \framesubtitle{The father}
  
  \begin{columns}
    
    \begin{column}{2.5cm}
      \pgfuseimage{lamport}
    \end{column}
    
    \begin{column}{6.5cm}
      \begin{block}{Leslie Lamport}
        \begin{itemize}
          \item Mathematician and Programmer
        \item Microsoft Research
        \end{itemize}
      \end{block}

      \begin{alertblock}{Motivation}
        \begin{itemize}
          \item Unsatisfaction with \TeX 80  macros writen by Max Diaz when he was writing a book
        \end{itemize}
      \end{alertblock}
    \end{column}
    
  \end{columns}
\end{frame}

\subsection{Basics}

\begin{frame}
  \frametitle{\LaTeX{} Macro}
  \framesubtitle{Example}

  \begin{block}{Macro}
    \bs documentclass[\alert<3>{a4paper}, \alert<4>{12pt}]\{\alert<2>{article}\}
  \end{block}
  
  \only<2>{
    \begin{alertblock}{Expansion:  $<$load file$>$}
      /usr/share/texmf-texlive/tex/latex/base/article.cls
    \end{alertblock}
  }

  \only<3>{
    \begin{alertblock}{Expansion: $<$set document size to A4 paper$>$}
      \bs DeclareOption\{a4paper\}\\
      \hspace{1cm} \{\bs setlength\bs paperheight \{297mm\}\% \\
      \hspace{1cm}  \bs setlength\bs paperwidth  \{210mm\}\}
    \end{alertblock}
  }
  \only<4>{
    \begin{alertblock}{Expansion: $<$set glyph to 12 point$>$}
        \bs DeclareOption\{12pt\}\{\bs renewcommand\bs @ptsize\{2\}\}
    \end{alertblock}
  }

\end{frame}

\subsection{Common Structure}

% create a grey scale to shadow some parts of text
\def\mygrey{black!10!bg}
\def\ntext{text text text}

\begin{frame}
  \frametitle{\LaTeX document: basic structure}
  \begin{tikzpicture}
    \draw (0,0) node [rectangle, %
      very thick,  %
      minimum size = 3mm,
      draw=black, 
      text width=10cm]{ 
      \bs documentclass[a4paper, 12pt]\{article\} \\
      { \color<1,6>{\mygrey} \alert<2>{\% preamble} } \\
      { \color<1>{\mygrey}  \alert<2>{\bs usepackage[parameter]\{package\}} } \\
      {  \color<1-2>{\mygrey}  \alert<3>{\bs begin\{document\}}  }\\
      { 
        \color<1-3>{\mygrey}
        \alert<4>{
        \hspace{0.25cm}\bs section\{First section name\} \\
        \hspace{0.5cm} \ntext \\
        } % end of alert
      } % end of color
      {
        \color<1-4>{\mygrey}
        \alert<5>{
          \small
          \hspace{0.75cm} \bs {subsection\{Second section name\}} \\
          \hspace{1.25cm}{\ntext} \\
          } % end of alert
        } % end of color
      {
        \color<1-5>{\mygrey}
        \alert<6>{
          \footnotesize
          \hspace{1.75cm} \bs {subsubsection\{Third section name\}} \\
          \hspace{2.25cm}{\ntext} \\
          } % end of alert
      }  % end of color
      { \color<1,2>{\mygrey}  \alert<3>{\bs end\{document\}}  }
};
\end{tikzpicture}
\end{frame}

\subsection{\sc{BIB}\TeX}

\def\spc{\hspace{2cm}}

\def\author{Leslie Lamport}
\def\title{LaTeX: A Document Preparation System}
\def\pub{Addison-Wesley}
\def\pyear{1986}
\def\producingtext{Producing Greek letters is as easy as}
\def\bibtexentry{
\small
@BOOK\{\alert<2-4>{latexbook}, \\
\hspace{2cm} AUTHOR = "\author", \\
\spc TITLE = "\title", \\
\spc  PUBLISHER = "\pub", \\
\spc  YEAR = \pyear \\
\}
}

\begin{frame}
  \frametitle{Bibliographic database}

  \begin{tikzpicture}
    \only<1-4>{
      \draw (0,0) node[fill=red,text width=10cm,text centered]
            {filename.bib};
  
            \draw  (0, -1.75) node[rectangle,
              draw=black,
              text width=10cm,
              text justified] {
              \bibtexentry
            };
    } % end of only
    \only<2-4>{
      \draw (0, -3.75) node[fill=blue!20!bg,text width=10cm,text centered]{filename.tex};
      \draw  (0, -4.325) node[rectangle,
        draw=black,
        text width=10cm,
        text justified] {
        \producingtext \$\bs pi\$\ \alert{$\tilde{ }$\ \bs cite\{latexbook\}}.
      };
    } % end of \only<2->

    \only<3,4>{
      \draw (0, -5.75) node[fill=black,text width=10cm,text justified]{
        \tt \small
        \color{white}
        \$ echo ``run bibtex and latex 2 times''\\
        \$ bibtex filename 
        \only<4> {
          \\
          \$ latex filename \\
          \$ latex filename
        }
      };
    }
    \only<5>{
      \draw  (0, 0) node[rectangle,
        draw=black!30!bg,
        text width=10cm,
        text justified] {
        \color{black!30!bg}{
          \bibtexentry
        } % end of draw
       };

      \draw (0, -2.25) node[fill=green!20!bg,text width=10cm,text centered]{manuscript}; % end of draw
      \draw  (0, -4) node[rectangle,
        draw=green!10!bg,
        text width=10cm,
        text justified] {
        \producingtext{} $\pi$ [1]. \\
        
        {\noindent \bf References} \\
        
        [1] \author. {\em \title}. \pub, \pyear.
      };
    } % end of \only<5>
    
  \end{tikzpicture}

\end{frame}


\begin{frame}
  \frametitle{``Three \LaTeX mistakes that people should stop  making?''}

  \begin{enumerate}
    
  \item \only<3>{``} Worrying too much about formatting and not
    enough about content;
    \only<2-> {
  \item Worrying too much about formatting and not
    enough about content;
    }
    \only<3> {
    \item Worrying too much about formatting and not
      enough about content.'' (Leslie Lamport)
    }
  \end{enumerate}

\end{frame}

\section{Further Details}

\begin{frame}
  \frametitle{\TeX 's children}
    \begin{list}
      {\color{blue!90!bg}{$\star$}}{\setlength{\parsep}{.5cm}}
      
    \item \href{ http://wiki.contextgarden.net/Main_Page}{C\sc{O}n\TeX t};
      \pause
    \item \href{http://www.pdftex.org/}{pdf\TeX}, pdf\LaTeX;
      \pause
    \item \href{http://omega.enstb.org/}{$\Omega$};
      \pause
        \item \href{http://scripts.sil.org/xetex}{X{\small \sc{E}}\TeX};
          \pause
          \item\href{http://www.luatex.org}{Lua\TeX}.
  \end{list}
\end{frame}


\begin{frame}
  \frametitle{References}

  \begin{thebibliography}{Haralambous, 2007}
  \bibitem[Knuth, 1991]{texbookcite}
    Donald Erwin Knuth
    \newblock {\em The \TeX book}.
    \newblock Addison-Wesley, 20th printing, 1991.

  \bibitem[Haralambous, 2007]{fontsencbookcite}
    Yannis Haralambous
    \newblock {\em Fonts \& encodings}
    \newblock O'Reilly Media, 1st edition, 2007.

  \bibitem[Lamport, 1994]{latexbookcite}
    Leslie Lamport
    \newblock {\em LaTeX: A Document Preparation System}
    \newblock Addison-Wesley, 2nd edition, 1994.

  \end{thebibliography}

\end{frame}

\begin{frame}
  \frametitle{Very useful URLs}
  
  \begin{list}
    {\color{darkblue}{$\bullet$}}{\setlength{\parsep}{.8cm}}

    \item \url{http://www.ctan.org}{ -- Comprehensive TeX Archive
      Network ( CTAN)};

    \item \url{http://www-cs-faculty.stanford.edu/~knuth/}{ -- Donald Knuth homepage};

      \item \url{http://www.lamport.org}{ -- Leslie Lamport homepage}.

  \end{list}

\end{frame}

\begin{frame}
  \frametitle{Where to find me?}
  \url{http://adrianoholanda.org}
\end{frame}

\end{document}
    
