\documentclass[a4paper, 12pt, ruledheader,pnumplain,normalfigtabnum]{abnt}

%% pacotes para portugues
\usepackage[utf8]{inputenc}
\usepackage[brazil]{babel}

% algumas definicoes
\def\etal{{\em at al.}}

\begin{document}

% CAPA 

\autor{José da Silva} 
\titulo{Proposta de uma Arquitetura
Interoperável para um Sistema de Informação em Saúde} 
\orientador{João  Souza} 
\instituicao{Universidade de São Pedro} \data{\today}
\local{Ribeirão Preto} 

\capa 

\folhaderosto

% estilo da pagina
\pagestyle{headings}

% inicializa contador de paginas
\setcounter{page}{1}

\sumario

\begin{resumo}
  Os problemas de interoperabilidade podem ser técnicos, onde os
  componentes de computação dos sistemas não permitem a cooperação
  devido às diferenças nos protocolos de comunicação ou semânticos,
  ocasionados devido à diversidade de representação da informação
  transmitida. Este trabalho propõe uma arquitetura para facilitar
  ambos os aspectos de interoperabilidade, sendo que a
  interoperabilidade técnica é proporcionada pela utilização de um
  {\em middleware} e a semântica, pela utilização de sistemas de
  terminologia adotados internacionalmente.
\end{resumo}

\clearpage

% paginas textuais c/ numeracao em arabico
\pagestyle{ruledheader}
\pagenumbering{arabic}
\setcounter{page}{1}

\chapter{Introdução}
\label{introducao}

Este Capítulo descreve a motivação para a pesquisa em padronização de
sistemas de informação em saúde . A Seção \ref{intro:interoperability}
descreve a importância da interoperabilidade entre os agentes de
saúde.  

\section{Interoperabilidade}
% referencia cruzada
\label{intro:interoperability}

Até o momento, nenhum sistema computadorizado de informação em saúde é
capaz de dar suporte a todo o espectro de conhecimento médico.
Conseqüentemente, hospitais e instituições de saúde adquirem sistemas
de diferentes fornecedores para satisfazer as suas necessidades
específicas. Como a troca de informações entre diferentes sistemas
geralmente não é trivial, esta prática acaba criando ``ilhas de
informação'' dentro das próprias instituições.  Considerando, por
exemplo, que um sistema computadorizado de registros clínicos deve
trabalhar com um sistema especialista de diagnóstico para melhorar o
atendimento ao paciente, para atingir integração ótima entre os dois
sistemas, a transferência dos registros de pacientes para o sistema
especialista deve ser automatizada. Na tentativa de atingir tal
objetivo, as diferenças entre os vocabulários controlados dos dois
sistemas foi citada por Wong \etal~\cite{wong:cbr:1994} como o maior
obstáculo para o intercâmbio de informações, mesmo se ambos os
sistemas tivessem sido implementados pelos mesmos desenvolvedores.

Um sistema computacional de arquitetura aberta em que houvesse a
possibilidade de se acoplar componentes de software que pudessem
colaborar na troca de informação com outros sistemas é o passo inicial
para ampliar o nível de acesso às informações do paciente. Porém, como
atualmente a informação médica trocada é fortemente baseada em
linguagem natural, há a necessidade de conformação dos termos usados a
algum vocabulário controlado para a representação dos conceitos. Com a
adoção destes procedimentos, há início da facilitação da troca de
informações entre sistemas heterogêneos criando assim condições para
alcançar a interoperabilidade.

A sintaxe e a semântica da troca de informações médicas utilizando uma
 linguagem específica para este fim podem ser padronizadas
utilizando um sistema de terminologia.  Os sistemas de terminologia
ajudam a lidar com a enorme variabilidade das expressões e termos
médicos, reduzindo a ambigüidade e relacionando os termos sinônimos.
Com a utilização destes sistemas, os termos médicos podem ser
representados por códigos que são independentes da linguagem natural.

\bibliographystyle{abnt-alf}
%\bibliographystyle{abnt-num}
\bibliography{myrefs}

\end{document}
